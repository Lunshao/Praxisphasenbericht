% !TEX root = vorlage.tex

\section{Theoretische Grundlagen}\label{section:theoretische_grundlagen}

\subsection{Aufgabenstellung}
Es soll ein \acl{PoC} erstellt werden um zu überprüfen, inwiefern die Implementierung eines
\acl{OS} \acl{fs} in das bestehende Projekt möglich ist.
Dafür müssen zuerst die grundsätzlichen Anforderungen an das \acl{fs} seitens des Projektes und Systems dokumentiert werden.
Danach findet eine testweise Einbindung des \acl{fs} in das bestehende Projekt statt.\\

Falls dies erfolgreich sein sollte, wird eine vollständige und saubere Implementierung mit samt
Dokumentation erfolgen.


\subsection{Ausgangslage}
Zu Beginn des Projekts gibt es das bereits bestehende System der \acl{IIP}, in welchem das \acl{fs} integriert werden soll.
Das vorhandene Betriebssystem ist dabei ein Integrity OS von Green Hills Software.


\subsection{Embedded filesystem littlefs}
Bei \acl{lfs} handelt es sich um ein \acl{OS} \acl{fs}, welches aus der embedded Welt kommt.
Häufig wird es auf kleinen Mikrocontrollern, wie einem ESP32 oder ähnlichen verwendet.\\

Auf den ersten Blick liefert es jedoch auch die von uns gewünschten Funktionen, welche das \acl{fs} besitzen muss.
Dazu gehören Eigenschaften, wie der Erhalt von Daten bei Stromverlust, gleichmäßiger Verschleiß des Arbeitsspeichers, sowie die Nutzung einer begrenzten Menge des Selbigen.\\

Da die sonstigen Anforderungen an die Funktionalitäten des \acl{fs} relativ gering sind,
scheint \acl{lfs} eine gute Wahl für die Umsetzung zu sein.


\subsection{Roadmap}\label{sec:roadmap}
Um den Start des Projekts zu vereinfachen und strukturieren, soll eine gewisse Roadmap entstehen,
welche die wichtigsten Punkte für das \acl{PoC} abzudecken.


\subsection{Planung der Arbeitspakete}
Die Planung der Arbeitspakete erfolgte erst nachdem ein Grundverständnis für das Projekt vorhanden war,
sowie ersichtlich ist, dass die Implementierung des \acl{fs} generell möglich ist.

Die Arbeitspakete werden in der Projektmanagement Software JIRA verwaltet.


\subsection{Dokumentation des Projekts}
Die Dokumentation des Projekts erfolgt parallel zum selbigen über die Wiki-Software Confluence.
Dies erfolgt in einem wöchentlichen Meeting, welches zum Austausch über den momentanen Projektstatus,
sowie die weitere Planung des Projekts dient.

Außerdem werden die einzelnen Arbeitspakete des Projektes ebenfalls im Confluence dokumentiert,
sowie eine abschließende Dokumentation des gesamten Projekts.

